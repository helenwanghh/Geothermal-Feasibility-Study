%%%%%%%%%%%%%%%%%%%%%%%%%%%%%%%%%%%%%%%%%%%%%%%%%%%%%%%%%%%%%%%%%%%%%%%%%%%%%%%%%%%%
% Do not alter this block (unless you're familiar with LaTeX
\documentclass{article}
\usepackage[margin=1in]{geometry} 
\usepackage{amsmath,amsthm,amssymb,amsfonts, fancyhdr, color, comment, graphicx, environ}
\usepackage{wrapfig}
\usepackage{xcolor}
\usepackage{mdframed}
\usepackage[shortlabels]{enumitem}
\usepackage{indentfirst}
\usepackage{hyperref}
\hypersetup{
    colorlinks=true,
    linkcolor=blue,
    filecolor=magenta,      
    urlcolor=blue,
}


\pagestyle{fancy}


% \newenvironment{problem}[2][Problem]
%     { \begin{mdframed}[backgroundcolor=gray!20] \textbf{#1 #2} \\}
%     {  \end{mdframed}}
\newenvironment{problem}[2][Problem]
    { \begin{mdframed}[backgroundcolor=white] \textbf{#1 #2} \\}
    {  \end{mdframed}}
% Define solution environment
\newenvironment{solution}
    {\textit{Solution:}}
    {}

\renewcommand{\qed}{\quad\qedsymbol}
\newcommand{\pder}[2]{\frac{\partial {#1}}{\partial {#2}}}
\newcommand{\pdder}[2]{\frac{\partial^2 {#1}}{\partial {#2}^2}}
\newcommand{\vd}[3]{{#1}_{{#2}}^{({#3})}}

% prevent line break in inline mode
\binoppenalty=\maxdimen
\relpenalty=\maxdimen

%%%%%%%%%%%%%%%%%%%%%%%%%%%%%%%%%%%%%%%%%%%%%
%Fill in the appropriate information below
\lhead{Your name: }
\rhead{ME 144} 
\chead{\textbf{Final Project 2022}}
%%%%%%%%%%%%%%%%%%%%%%%%%%%%%%%%%%%%%%%%%%%%%

\begin{document}

\begin{mdframed}[backgroundcolor=blue!20]
Please submit your solution as the downloaded pdf of this document.
\end{mdframed}

\begin{problem}{1}
% \textbf{[10pts]} 

The goal of this document is to derive the set of equations that must be solved for the model of the room based on the following assumptions:
\begin{itemize}
    \item The temperature of the roof $T_0(t)$ (surface exposed to the outside) is uniform across the surface area of the roof.
    \item The temperature of the ceiling $T_1(t)$ (surface of the roof exposed to the room) is uniform across the surface area of the ceiling.
    \item The temperature of the room $T_2(t)$ is defined as its average over the volume of the room $V_{room}$
    \begin{equation*}
        T_2(t)=\frac{1}{V_{room}}\iiint_{V_{room}}T(t,x,y,z)dxdydz
    \end{equation*}
    \item In the conservation of energy for the room, the thermodynamic properties of the air are evaluated at $T_2(t)$ everywhere throughout the volume of the room
\end{itemize}
Based on the conservation of energy applied to a control surface $CS$
\begin{equation}
    \iint_{CS}\rho C_p \frac{dT}{dt}dS=\sum E_{in}-\sum E_{out}+ \sum E_g
\end{equation}
or a control volume $CV$
\begin{equation}
    \iiint_{Cv}\rho C_p \frac{dT}{dt}dV=\sum E_{in}-\sum E_{out}+ \sum E_g
\end{equation}
derive the set of equations for ($T_0,T_1,T_2$). For fluxes and heat rates, write them in a compact form such as, $q_{weather}(T_{atm},T_{sky},T_0$ or $q''_{cond}(T_0,T_1)$. \textbf{Your choice of heat flux or heat rate formulation should be consistent with your equations.}
\end{problem}
\begin{solution}
\subsection*{Brief explanation of the different control surface used and why}
Your text here 

\subsection*{Final equations}
\begin{eqnarray}
\frac{dT_0}{dt}&=&...\\
\frac{dT_1}{dt}&=&...\\
\frac{dT_2}{dt}&=&...
\end{eqnarray}

\end{solution}


\end{document}